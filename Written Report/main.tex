%%%%%%%%%%%%%%%%%%%%%%%%%%%%%%%%%%%%%%%%%%%%%%%
%%% Template for lab reports used at STIMA
%%%%%%%%%%%%%%%%%%%%%%%%%%%%%%%%%%%%%%%%%%%%%%%

%%%%%%%%%%%%%%%%%%%%%%%%%%%%%% Sets the document class for the document
% Openany is added to remove the book style of starting every new chapter on an odd page (not needed for reports)
\documentclass[10pt,english, openany]{book}

%%%%%%%%%%%%%%%%%%%%%%%%%%%%%% Loading packages that alter the style
\usepackage[]{graphicx}
\usepackage[]{color}
\usepackage{alltt}
\usepackage[T1]{fontenc}
\usepackage[utf8]{inputenc}
\setcounter{secnumdepth}{3}
\setcounter{tocdepth}{3}
\setlength{\parskip}{\smallskipamount}
\setlength{\parindent}{0pt}


% Set page margins
\usepackage[top=100pt,bottom=100pt,left=68pt,right=66pt]{geometry}

% Package used for placeholder text
\usepackage{lipsum}

% Prevents LaTeX from filling out a page to the bottom
\raggedbottom

% Adding both languages
\usepackage[english]{babel}

% All page numbers positioned at the bottom of the page
\usepackage{fancyhdr}
\fancyhf{} % clear all header and footers
\fancyfoot[C]{\thepage}
\renewcommand{\headrulewidth}{0pt} % remove the header rule
\pagestyle{fancy}

% Changes the style of chapter headings
\usepackage{titlesec}
\titleformat{\chapter}
   {\normalfont\LARGE\bfseries}{\thechapter.}{1em}{}
% Change distance between chapter header and text
\titlespacing{\chapter}{0pt}{50pt}{2\baselineskip}

% Adds table captions above the table per default
\usepackage{float}
\floatstyle{plaintop}
\restylefloat{table}

% Adds space between caption and table
\usepackage[tableposition=top]{caption}

% Adds hyperlinks to references and ToC
\usepackage{hyperref}
\hypersetup{hidelinks,linkcolor = black} % Changes the link color to black and hides the hideous red border that usually is created

% If multiple images are to be added, a folder (path) with all the images can be added here 
\graphicspath{ {Figures/} }

% Separates the first part of the report/thesis in Roman numerals
\frontmatter


%%%%%%%%%%%%%%%%%%%%%%%%%%%%%% Starts the document
\let\cleardoublepage=\clearpage
\begin{document}

%%% Selects the language to be used for the first couple of pages
\selectlanguage{english}

%%%%% Adds the title page
\begin{titlepage}
	\clearpage\thispagestyle{empty}
	\centering
	\vspace{1cm}

	% Titles
	% Information about the University
	{\normalsize CM50304: AI Challenge \\ 
		Department of Computer Science \\
		University of Bath, UK \par}
		\vspace{3cm}
	{\Huge \textbf{Air quality monitoring using an AI-based device}} \\
	%\vspace{1cm}
	%{\large \textbf{xxxxx} \par}
	\vspace{4cm}
	{\normalsize FIRST LAST \\ % \\ specifies a new line
	             FIRST LAST \\
	             FIRST LAST \\
	             FIRST LAST \\
	             FIRST LAST \par}
	\vspace{5cm}
    
    \centering \includegraphics[scale=0.1]{University_of_Bath_logo.png}
    
    \vspace{0.5cm}
		
	% Set the date
	{\normalsize 8th May 2020 \par}
	
	\pagebreak

\end{titlepage}

% Adds a table of contents
\tableofcontents{}

%%%%%%%%%%%%%%%%%%%%%%%%%%%%%%%%%%%%%%%%%%%%%%%%%%%%%%%%%%%%%%%%%%%%%%%%%%%%%%%%%%%%%%%%%%%%
%%%%%%%%%%%%%%%%%%%%%%%%%%%%%%%%%%%%%%%%%%%%%%%%%%%%%%%%%%%%%%%%%%%%%%%%%%%%%%%%%%%%%%%%%%%%
%%%%% Text body starts here!
\mainmatter

\chapter{Summary}\label{chapt:sum}
[\textit{Briefly describe the goal of the project.}]

\chapter{Problem definition and background}
\section{Problem specification}
[\textit{This section should state the challenge, motivate why this challenge is important and explain how it relates to AI.}]

\section{Literature review}
[\textit{This section should review the existing solutions for this challenge and critically analyse ethical issues related to them. It is maybe the case that there are no existing solutions for the chosen challenge. In this case this should be clearly stated and analysed why it is so. If applicable, existing solutions for related challenges may be reviewed here. For example, if your solution is based on solutions for other challenges you should review and critically analyse them in this section.}]

\section{Proposed solution}
[\textit{This section should explain your ambitious solution and why it requires or why it benefits from the interdisciplinary approach. If the solution involves novel research the section should explain why you believe that this research is likely to be successful in the foreseeable future (see above).}]

\chapter{Air quality monitoring device}\label{chapt:doe}
[\textit{Describe the engineering part of the project.}]

\section{Hardware}
\section{Software}
\section{Prototype Description}
[\textit{This section should explain the scope of your developed prototype, explain how it works and how it relates to the whole solution. It should clearly demonstrate the interdisciplinary effort in this development. If interdisciplinary development was not possible (see above) it should be clearly stated in this section and explain why it was not possible to exploit some of the expertise from members of your group. This section should also reflect how the experience of developing this prototype has confirmed or questioned your whole solution and how it may have changed your view on the whole solution or possibly the challenge in general.}]

\chapter{AI model}\label{chapt:model}
[\textit{Describe the AI model in the project}]
\section{Datasets}
\section{Methodology}
\section{Results}
\section{Etc}

\chapter{Social implications}
%\textit{Social science section 
%
%Environmental justice (EJ) emerged in the early 1980s as a frame for understanding iniquities in protection from environmental costs and hazards (Ottinger et al. 2017). Increasingly, however, the EJ research agenda is concerned not just with distributive justice but with procedural or participatory justice, as well as recognition and capabilities (Schlosberg 2007). We have applied the lens of environmental justice to the design of our protype and network of sensors which will generate largescale datasets which will be used to analyse trends, anomalies (etc. – explain).  This will in turn be part of a wider intervention aimed at enabling public participation in scientific practice, deliberation and policy making, and at enhancing accountability in environmental governance, and shaping new opportunities to address environmental injustices. 
%
%Existing accountability tools: 
%
%The Convention on Access to Information, Public Participation in Decision-making and Access to Justice in Environmental Matters (Aarhus Convention) establishes a number of rights (for individuals and associations) with regard to the environment, including: access to environmental information, public participation in environmental decision-making, and a right to hold public authorities accountable for failure to respect rights and obligations under environmental law. 
%
%%https://www.ucl.ac.uk/european-institute/sites/european-institute/files/next_generation_environmental_law_paper_final.pdf
%
% more here 
%
%Air quality as environmental justice
%
%Mitchell and Dorling’s 2003 research into environmental injustices in road traffic-related air pollution in the the UK showed that communities that are most polluted, and which emit the least pollution, tend to be amongst the poorest. In their recent follow-up study, Barnes et al. (2019) found these correlations to be even stronger (see Figure x and Figure y). Moreover, an earlier assumption that poor people contribute substantially to the emissions to which they were exposed (by owning older, more polluting vehicles) was shown to be unfounded.
%
% 
%Figure x Barnes et al. 
% 
%
%Figure y Barnes et al. 
%
%
%When considering the public policy dimensions of their findings, the authors recommended that policy interventions — to address both the underlying air quality problems in the UK, and the highly unequal risks of exposure — ought not to rely predominantly on technological improvements and individual choices. With this in mind, we approach our prototype design alert to the limitations of interventions which focus on individual behaviour change but do not account for wider social dimensions e.g. decisions in land use, economic development and transport planning. This is consistent with the EJ perspective which sees behavioural changes as desirable but insufficient to solve the health problems caused by air pollution (Maturo and Moretti 2018). Shove (2010) similarly argues for UK environmental policy to move away from a model of social change based on ‘ABC’ — attitude, behaviour, and choice — towards more systemic and sociologically informed analyses. 
%
%The relationship between power, inequality and the social construction of environmental problems! Bickerstaff highlights how experience of pollution is related to place - people sometimes ‘Other’ pollution, particularly where they feel disempowered.
%
%
%“The all-too-abstract scientific constructions of air pollution lack any relevance to most people, not only to their culturally embedded understandings but also to the practices by which air pollution science is produced.”
%
%With the system featuring computational techniques that most people will not be familiar with, it is important that the system is not just transparent but, to some degree, explainable (talk about different levels of exploitability). 
%
% One way of addressing this might be to intwine scientific analysis with more deliberative practice. 
%
%Citizen science 
%
%As awareness of ambient air pollution and its negative effects on health has grown, some groups of citizens have engaged in citizen science initiatives to measure and hold authorities accountable for air quality. Citizen science is broadly defined as ‘science developed and enacted by the citizens themselves’ (Irwin 1995). In the context of air quality, citizens have participated in initiatives which use low-cost measuring devices to supplement official measurements (EEA 2019). 
%
%
%Methodology 
%
%If our prototype were to be developed fully, a scoping and feasibility study would be undertaken to identify and understand issues including: 
%
%•	how people would interact with the devices;
%•	how to ensure participation of marginalised groups (recognition and capabilities?);
%•	the most appropriate mechanisms to use the analyses for public accountability; 
%•	
%
%This would also include an options appraisal of possible funding and ownership models. Currently, air quality monitoring sensors in the UK are owned and operated by Defra and local authorities. However, alternative and community based models of ownership may be more desirable, and could address the risk of co-option which (author) identifies in citizen science inititaives. 
%
%References 
%
%EEA. 2019.  
%
%Irwin, A. 1995. 
%
%Maturo A., Moretti V. (2018) Sociological Theories on Air Pollution: Between Environmental Justice and the Risk Society Approach. In: Capello F., Gaddi A. (eds) Clinical Handbook of Air Pollution-Related Diseases. Springer, Cham 
%
%Mitchell, G. and Dorling, D. 2003. An environmental justice analysis of British air quality. Environment and Planning A 2003, volume 35, pages 909- 929 
%
%Ottinger, G., 
%
%Schlosberg 
%
%Shove, Beyond the ABC: climate change policy and theories of social change
%}

\chapter{Discussion}\label{chapt:results}
\section{Critical reflection of proposed solution}
[\textit{This section should provide an analysis of how the proposed solution meets accountability, responsibility, transparency and other ethical concerns. It should honestly reflect both the positive and negative impact the solution may have. If the view from the ethical perspective on the solution has changed during the work on this project this section should describe that and explain why the view has changed.}]

\section{Future work}
[\textit{This section should describe immediate and long-term steps after development of the prototype that should be made to achieve the whole solution. If during development of prototype or critical reflection of the solution you come to the conclusion that the proposed solution is not useful or not ethically responsible or the overall challenge should not be formulated in this way (see above), this section should clearly state this case and may omit description of future work.}]

\section{Teamwork process}
[\textit{This section should describe how your group organised teamwork, what communication channels and collaboration tools you used, how you distributed the tasks, what process allowed you to learn from another discipline during this project.}]

\section{Lessons learnt (project)}
[\textit{This section should reflect on the whole project from the teamwork perspective: what was done well, what could have been improved, what would you change if you were to work on another project in this group, what could you recommend for an interdisciplinary team.}]

\section{Lessons learnt (per discipline)}
[\textit{Each member of the group should state what they learnt from another discipline during this project.}]
\subsection{Person 1}
\subsection{Person 2}
\subsection{Person 3}
\subsection{Person 4}
\subsection{Person 5}

\chapter{Conclusions}

\chapter{Individual contributions}
[\textit{This section should briefly state individual contributions from each team member.}]

\pagebreak


% Adding a bibliography if citations are used in the report
%\bibliographystyle{plain}
%\bibliography{bibliography}
% Adds reference to the Bibliography in the ToC
\addcontentsline{toc}{chapter}{\bibname}

\pagebreak

\chapter*{Appendix A: Code Etc}
% \section{Reference solution data}


\end{document}
